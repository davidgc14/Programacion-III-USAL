\documentclass[fleqn]{article}

%\pgfplotsset{compat=1.17}

\usepackage{mathexam}
\usepackage{amsmath}
\usepackage{amsfonts}
\usepackage{graphicx}
\usepackage{systeme}
\usepackage{microtype}
\usepackage{multirow}
\usepackage{pgfplots}
\usepackage{listings}
\usepackage{tikz}
\usepackage{dsfont} %Numeros reales, naturales...
\usepackage{cancel}
\usepackage{setspace} %Para modificar el interlineado
\usepackage[spanish]{babel}

\usepackage{pgfplots}
\pgfplotsset{compat=1.7}

%\graphicspath{{images/}}
\newcommand*{\QED}{\hfill\ensuremath{\square}}


%Estructura de ecuaciones
%\setlength{\textwidth}{15cm} \setlength{\oddsidemargin}{5mm}
%\setlength{\textheight}{23cm} \setlength{\topmargin}{-1cm}



\author{David García Curbelo}
\title{Topología}

\pagestyle{empty}

%\renewcommand{\baselinestretch}{2} para modificar el interlineado en todo el doc

\def\R{\mathds{R}}
\def\Z{\mathds{Z}}
\def\N{\mathds{N}}
\def\S{\mathds{S}}

\def\sup{$^2$}

\def\next{\quad \Rightarrow \quad}

\begin{document}
    \doublespace

    \setcounter{page}{1}
    \pagestyle{plain}

    \begin{center}
        {\Large\bf{IMBD21}} \\
        {\large\bf{Informe del proyecto}} \\
        \bf{David García Curbelo}\\
        
    \end{center}

    El presente proyecto tiene como objetivo facilitar al usuario la lectura, escritura y almacenamiento de los datos,
    en este caso concreto procedentes de películas, con sus respectivos actores, directores, géneros, etc.

    Presentando una arquitectura MVC, han sido añadidos al paquete data los módulos habituales, y los necesarios para 
    el correcto funcionamiento de la aplicación: Filmoteca, Pelicula, Actor y Director. Además, en el paquete view me he servido 
    de un ejemplar auxiliar al archivo View, con la intención de que éste sólo controle la interfaz de los menús principales
    y el programa de arranque e importación de archivos. El visor auxiliar (ViewAux) se encarga de controlar las funciones de cada 
    uno de los submenús que se encuentran en la clase view. De este modo la revisión de código queda más organizada y de más fácil acceso.\\
    
    Se han aportado a nivel general un alto contenido de mensajes de error, para que el usuario pueda identificar no sólo los errores sino 
    también su ubicación para la posterior corrección. Así mismo, y con idéntica finalidad, se han añadido mensajes de información al inicio del programa,
    para informar acerca de los archivos que han sido importados y el estado de éstos. 
    
    Por ello he añadido, en un comentario, una parada temporal en el programa inicial (runMenu) para que el usuario pueda ver los mensajes de información 
    referentes a la importación de archivos que se acaba de realizar.
    Dicha parada no es estrictamente necesaria (además, no está en funcionamiento. Se encuentra comentada en la linea 26 de View.java), 
    y se puede eliminar en el futuro junto con el clear() que se ejecuta cada vez que se encuentra 
    en el bucle while del runMenu en la clase View. Dicha parada ha requerido un throws InterruptedException en el runMenu (también en un comentario en la línea
    8) para informar de que se dará lugar una parada temporal de 5000 milisegundos, que ha sido ejecutada mediante Thread.sleep(time) de la línea 26.

    Se ha procurado un uso máximo de biblioteca.jar, mediante la cual ha sido usado el método printToScreen3() en la clase ViewAux para la impresión
    de listas requeridas para el proyecto.

    La importación de archivos ha sido estructurada de tal forma que se dé prioridad al archivo de tipo binario, y en caso de que éste no exista, 
    o que se pueda encontrar dañado, se intentará la importación del archivo de texto (para priorizar el funcionamiento del programa antes que la importación
    de archivos binarios). En caso de que no se pueda encontrar ninguno de los archivos, se finalizará la ejecución del programa y se mostrará un mensaje por cada 
    importación realizada, y se indicará qué archivo se ha importado de cada tipo, y cuál de ellos no se ha podido importar (con su respectivo motivo).





\end{document}